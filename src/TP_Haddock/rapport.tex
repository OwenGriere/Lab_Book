\documentclass[10pt,a4paper]{report}

% ========= Préambule importé depuis help/base.tex (extrait avant \begin{document}) =========

% =============== Encodage & langue ===============
\usepackage[T1]{fontenc}
\usepackage[utf8]{inputenc}
\usepackage[french]{babel}
\usepackage{lmodern}
\usepackage{amsmath}
\usepackage{microtype}
\usepackage{multicol}
\usepackage{float} 

% =============== Mise en page ===============
\usepackage[a4paper,margin=1.5cm, bottom=2.2cm]{geometry}
\usepackage{setspace}
\setlength{\headheight}{15pt}
\setlength{\footskip}{12mm}  
\onehalfspacing

% =============== Images & couleurs ===============
\usepackage{graphicx}
\graphicspath{{figures/}}
\usepackage[dvipsnames]{xcolor}
\usepackage{amsmath,amssymb,amsfonts}

\usepackage{tikz}
\DeclareRobustCommand{\legenddot}[2][3pt]{%
  \tikz[baseline=-0.6ex]{\draw[draw=black,fill=#2] (0,0) circle (#1);}%
}
% =============== En-têtes / pieds de page ===============
\usepackage{fancyhdr}
\pagestyle{fancy}
\fancyhf{}
\renewcommand{\headrulewidth}{0pt}
\lhead{} \chead{} \rhead{}          
\cfoot{\thepage}

% =============== Liens cliquables & signets PDF ===============
\usepackage{hyperref}
\hypersetup{
  pdfauthor   = {Owen GRIERE},
  pdftitle    = {Rapport de Stage - CRCT - 2025},
  pdfsubject  = {Rapport de Stage},
  pdfcreator  = {LaTeX},
  colorlinks  = true,
  linkcolor   = teal,
  urlcolor    = teal,
  citecolor   = teal
}
\usepackage{bookmark}
\usepackage{wrapfig}
\usepackage{array}
\usepackage{enumitem}
\usepackage{tabularx}
\usepackage{etoolbox} 
\usepackage{caption, subcaption}
\captionsetup{font={scriptsize,it}}

%=============== Bibliographie ===============

\makeatletter
\newcommand{\tocnopage}[2]{%
  \@ifpackageloaded{hyperref}{%
    \addtocontents{toc}{\protect\contentsline{#1}{#2}{}{}}
  }{%
    \addtocontents{toc}{\protect\contentsline{#1}{#2}{}}
  }%
}
\usepackage[section]{placeins}
\usepackage{titlesec}
\usepackage[backend=biber,style=numeric]{biblatex}
\addbibresource{references/references.bib}

%=============== Glossaire ===============

\usepackage[acronym]{glossaries}
\makeglossaries
\setacronymstyle{long-short}

%=============== Format des titres ===============

\makeatother
\setlist{nosep}

\titleformat{\chapter}{\normalfont\huge\bfseries}{\thechapter}{1em}{}
\titlespacing*{\chapter}{0pt}{-10pt}{20pt}

\titleformat{\section}[block]{\Large\bfseries}{\thesection}{0.8em}{}      
\titlespacing*{\section}{0pt}{2ex plus .2ex}{1ex plus .1ex}               

\titleformat{\subsection}[block]{\large\bfseries}{\thesubsection}{0.8em}{} 
\titlespacing*{\subsection}{0pt}{1.5ex plus .2ex}{0.7ex plus .1ex}


%=======================================
%============ Page de garde ============
%=======================================


\begin{document}


\begin{titlepage}
  \centering
  {\Large \textbf{Rapport de TP : Haddock}}\par
  \vspace{1cm}
  {\large Simulation Haddock d'une interface Protéine-Protéine}\par
  \vfill
  {\large Auteur : Owen GRIERE}\par
  {\large Date : \today}\par
\end{titlepage}


\tableofcontents
\pdfbookmark[section]{Sommaire}{toc}
\newpage







\chapter{Inspections des deux protéines et de leurs structures résolues}

\section{Inspection de la protéine EA2}

\textbf{Quelle est la méthode expérimentale utilisée pour résoudre la structure et quelle
est la résolution de cette structure ? Est-ce que la structure possède un groupement phosphate
dans la structure ?}

La méthode de expérimentale utilisé pour résoudre la structure est X-Ray avec 2.10 Angstrom de résolution 

    Method: X-RAY DIFFRACTION
    Resolution: 2.10 Å
    R-Value Observed: 0.162 (Depositor) 

\noindent En relevant les histidines on peut voir ceci:

\begin{figure}[H]
  \centering
  \includegraphics[width=0.63\textwidth]{with_hist.png}
  \caption{Visualisation de la protéine EA2 sur PyMol en révélant les Histidines}
  \label{fig:hist}
\end{figure}

\noindent De plus cette protéine ne présente pas de groupement phosphate.

\textbf{Est-ce que la surface identifiée est continue ? Est-ce que cela peut former un patch d’interaction avec une protéine ?}

\begin{figure}[H]
  \centering
  \includegraphics[width=0.63\textwidth]{ea2.png}
  \caption{Surface possible d'interaction protéine-protéine}
  \label{fig:surface}
\end{figure}

\noindent La surface semble continue est orienté, elle peut correspondre a une interface d'intéraction

\section{Inspection de la protéine HPr}

\textbf{Que peut-on remarquer lorsque la structure est affichée sur PyMol ?}

\noindent On remarque que les feuillets Beta forme un surface qui peut etre une interface d'intération avec une autre macromolécule

\textbf{est-ce que la surface identifiée est continue ? Est-ce que cela peut former un patch
d’interaction avec une protéine ? }

\noindent La surface d'intéraction est pour le coup complétement continue est peut servir de clé dans une intéraction serure-clé lors du docking de 2 macromolécules.

\textbf{Que peut-on dire des conformations prises par les résidus identifiés par l’étude RMN, en rouge ?}

\noindent On remarque que malgré 30 conformation différentes le site d'intéractions semble ne pas changer drastiquement par rapport a la géometrie de la protéine.

\begin{figure}[H]
  \centering
  \includegraphics[width=0.63\textwidth]{hpr.png}
  \caption{Surface possible d'interaction protéine-protéine pour la protéine HPR, regroupant 30 conformations différentes}
  \label{fig:hpr}
\end{figure}








\chapter{Modification de la structure de EA2 avec phosphorylation}

\textbf{Consultez la liste des acides aminés modifiés pris en charge. Quel est le nom de résidu approprié pour une phospho-histidine dans HADDOCK ?}

\noindent Afin de Phosporiser une histidine nous utilisons la modifications de résidus par Haddock en utilisant la modificatin de résidus suivantes

    NEP: NE phosphorylated HIS.
    Atoms:
    N,HN,CA,HA,CB,HB1,HB2,CG,ND1,HD1,CD2,HD2,CE1,HE1,NE2,P,O1P,O2P,O3P,C,O









\chapter{Simulation de docking des protéines EA2 et HPr}

Cette partie n'est pas faisable a cause du temps d'execution de la simulation

\section{Soumission et validation des structures}

\section{Définition des contraintes}

\section{Vérification de l'état de protonation des histidines}

\section{Soumission du calcul}






\chapter{Analyse des résultats}

6 Clusters on été généré par cette analyse voici leur energie associées : \\

\begin{itemize}[label=o]
  \item CLuster 1:
  \begin{itemize}[label=--]
    \item Van der Waals energy	-30.4 +/- 3.2
    \item Electrostatic energy	-604.5 +/- 41.4
    \item Desolvation energy	-14.4 +/- 1.1
    \item Restraints violation energy	18.3 +/- 8.6
  \end{itemize}

  \item Cluster 2 :
  \begin{itemize}[label=--]
    \item Van der Waals energy	-34.0 +/- 3.1
    \item Electrostatic energy	-385.9 +/- 18.9
    \item Desolvation energy	-14.1 +/- 1.8
    \item Restraints violation energy	16.5 +/- 10.4
  \end{itemize}

  \item Cluster 3 :
  \begin{itemize}[label=--]
    \item Van der Waals energy	-27.5 +/- 6.5
    \item Electrostatic energy	-440.9 +/- 12.9
    \item Desolvation energy	-11.8 +/- 2.3
    \item Restraints violation energy	25.5 +/- 12.9
  \end{itemize}

  \item Cluster 4 :
  \begin{itemize}[label=--]
    \item Van der Waals energy	-20.1 +/- 4.5
    \item Electrostatic energy	-420.9 +/- 45.9
    \item Desolvation energy	-9.1 +/- 2.4
    \item Restraints violation energy	48.6 +/- 43.7
  \end{itemize}

  \item Cluster 5 :
  \begin{itemize}[label=--]
    \item Van der Waals energy	-32.6 +/- 8.8
    \item Electrostatic energy	-337.4 +/- 55.4
    \item Desolvation energy	-9.7 +/- 3.9
    \item Restraints violation energy	30.5 +/- 25.2
  \end{itemize}

  \item Cluster 6 :
  \begin{itemize}[label=--]
    \item Van der Waals energy	-14.5 +/- 4.4
    \item Electrostatic energy	-361.4 +/- 20.0
    \item Desolvation energy	-14.2 +/- 3.6
    \item Restraints violation energy	49.3 +/- 28.5
  \end{itemize}
\end{itemize}


\noindent \\ Pour chacun des Cluster on calcul:
\[
HADDOCK_score = 1,0 * Evdw + 0,2 * Eelec + 1,0 * Edesol + 0,1 * Eair
\]


le \(HADDOCK_score\) du Cluster 1 (meilleur classement) est de -163.9 +/- 9.3 tandis que le \(HADDOCK_score\) du Cluster 2,
donc le second mieux classé, est de -124.9 +/- 2.5. Ainsi, on peut en déduire que le meilleur cluster est significativement meilleur 
que les autres clusters. 

\chapter{Visualisation et interprétation des résultats}


\begin{figure}[H]
  \centering
  \includegraphics[width=0.63\textwidth]{allcluster.png}
  \caption{Ensemble des cluster de cette interaction aligné sur le premier cluster}
  \label{fig:all}
\end{figure}

Les différentes position semble assez similaire.

\begin{figure}[H]
  \centering
  \includegraphics[width=0.63\textwidth]{interface.png}
  \caption{Visualisation de l'interface entre EA2 et HPR avec les meilleurs conformation des 6 Clusters}
  \label{fig:interface}
\end{figure}

\noindent On remarque que l'on retrouve bien les résidus présent dans les surface d'intractions en contact les uns avec les autres 
dans ces conformations

\begin{figure}[H]
  \centering
  \includegraphics[width=0.63\textwidth]{cluster1.png}
  \caption{Visualisation de l'interface entre EA2 et HPR avec uniquement le cluster 1}
  \label{fig:interface}
\end{figure}

\end{document}
