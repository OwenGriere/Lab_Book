\documentclass[12pt,a4paper]{report}

% ========= Préambule importé depuis help/base.tex (extrait avant \begin{document}) =========

% =============== Encodage & langue ===============
\usepackage[T1]{fontenc}
\usepackage[utf8]{inputenc}
\usepackage[french]{babel}
\usepackage{lmodern}
\usepackage{amsmath}
\usepackage{microtype}
\usepackage{multicol}
\usepackage{float} 

% =============== Mise en page ===============
\usepackage[a4paper,margin=1.5cm, bottom=2.2cm]{geometry}
\usepackage{setspace}
\setlength{\headheight}{15pt}
\setlength{\footskip}{12mm}  
\onehalfspacing

% =============== Images & couleurs ===============
\usepackage{graphicx}
\graphicspath{{figures/}}
\usepackage[dvipsnames]{xcolor}
\usepackage{amsmath,amssymb,amsfonts}

\usepackage{tikz}
\DeclareRobustCommand{\legenddot}[2][3pt]{%
  \tikz[baseline=-0.6ex]{\draw[draw=black,fill=#2] (0,0) circle (#1);}%
}
% =============== En-têtes / pieds de page ===============
\usepackage{fancyhdr}
\pagestyle{fancy}
\fancyhf{}
\renewcommand{\headrulewidth}{0pt}
\lhead{} \chead{} \rhead{}          
\cfoot{\thepage}

% =============== Liens cliquables & signets PDF ===============
\usepackage{hyperref}
\hypersetup{
  pdfauthor   = {Owen GRIERE},
  pdftitle    = {Rapport de Stage - CRCT - 2025},
  pdfsubject  = {Rapport de Stage},
  pdfcreator  = {LaTeX},
  colorlinks  = true,
  linkcolor   = teal,
  urlcolor    = teal,
  citecolor   = teal
}
\usepackage{bookmark}
\usepackage{wrapfig}
\usepackage{array}
\usepackage{enumitem}
\usepackage{tabularx}
\usepackage{etoolbox} 
\usepackage{caption, subcaption}
\captionsetup{font={scriptsize,it}}

%=============== Bibliographie ===============

\makeatletter
\newcommand{\tocnopage}[2]{%
  \@ifpackageloaded{hyperref}{%
    \addtocontents{toc}{\protect\contentsline{#1}{#2}{}{}}
  }{%
    \addtocontents{toc}{\protect\contentsline{#1}{#2}{}}
  }%
}
\usepackage[section]{placeins}
\usepackage{titlesec}
\usepackage[backend=biber,style=numeric]{biblatex}
\addbibresource{references/references.bib}

%=============== Glossaire ===============

\usepackage[acronym]{glossaries}
\makeglossaries
\setacronymstyle{long-short}

%=============== Format des titres ===============

\makeatother
\setlist{nosep}

\titleformat{\chapter}{\normalfont\huge\bfseries}{\thechapter}{1em}{}
\titlespacing*{\chapter}{0pt}{-10pt}{20pt}

\titleformat{\section}[block]{\Large\bfseries}{\thesection}{0.8em}{}      
\titlespacing*{\section}{0pt}{2ex plus .2ex}{1ex plus .1ex}               

\titleformat{\subsection}[block]{\large\bfseries}{\thesubsection}{0.8em}{} 
\titlespacing*{\subsection}{0pt}{1.5ex plus .2ex}{0.7ex plus .1ex}


%=======================================
%============ Page de garde ============
%=======================================


\begin{document}


\begin{titlepage}
  % --- Bandeau de logos en haut de page ---
  \begin{center}
    \setlength{\tabcolsep}{8pt} % espace horizontal entre logos
    \renewcommand{\arraystretch}{1.0} % hauteur des lignes
    \begin{tabular*}{\textwidth}{@{\extracolsep{\fill}}ccccccc@{}}
      \includegraphics[height=3.5cm]{logo_eidd.png} \\
    \end{tabular*}
  \end{center}

  \vspace*{3cm} % espace après les logos

  % --- Titre principal ---
  \centering
  {\fontsize{44}{52}\selectfont\bfseries
  TP Modélisation MD Gros Grain\par}

  \vspace*{\fill}

  % --- Ligne et informations ---
  \noindent\rule{\textwidth}{1pt}
  \vspace{0.6em}
  {\raggedright
    \large Auteur : Owen GRIERE\par
    \large Date : \today\par
  }

\end{titlepage}


\cleardoublepage
\pdfbookmark[section]{Sommaire}{toc} 
\tableofcontents
\newpage

% ====================================================================================
% =================================== Introduction ===================================
% ====================================================================================
\chapter{Introduction}

\begin{multicols}{2}
  La simulation MD est une méthode numérique permettant de suivre le mouvement de particules en résolvant les équations de Newton 
  à partir de forces dérivées de potentiels d’interaction. Dans le cas d’une modélisation gros grain (coarse-grain), plusieurs atomes sont regroupés en 
  une seule entité appelée « beads », afin de simplifier la description du système. Cette approche réduit considérablement le nombre de 
  degrés de liberté et donc le coût computationnel. Elle permet ainsi d’accéder à des échelles de temps et d’espace plus grandes que celles atteignables par 
  les simulations atomistiques classiques. Les interactions entre les billes sont modélisées par des potentiels effectifs, ajustés pour reproduire les propriétés 
  physiques globales du système réel. Ce type de simulation est particulièrement utilisé pour étudier les macromoldécules biologiques comme les protéines où les 
  détails atomiques ne sont pas systématiquement indispensables. La MD gros grain constitue donc un compromis entre réalisme structurel et efficacité numérique.
  Dans notre cas nous étudions différentes conformation d'un ARN grâce à diverses structures crystalisées (PDB). L'objectif final étant de trouver la conformation 
  la plus stable de l'ARN. Cette recherche dépendra du mapping (regroupement de groupes d'atomes en beads pour obtenir une simulation gros grain) que nous choisirons,
  d'un champ de force spécifique a notre problème. Une fois les étapes précédentes accomplies nous utiliserons alors une methode de Monte Carlo pour explorer les diverses 
  conformations jusqu'à obtenir la conformation la plus stable, donc la plus basse en énergie. 
\end{multicols}

\vspace{1em}

\begin{multicols}{2}
  MD simulation is a numerical method for tracking particle motion by solving Newton's equations 
  based on forces derived from interaction potentials. In coarse-grain modeling, several atoms are grouped together into 
  a single entity called “beads” in order to simplify the description of the system. This approach significantly reduces the number of 
  degrees of freedom and therefore the computational cost. It thus allows access to larger time and space scales than those achievable by 
  classical atomistic simulations. The interactions between the beads are modeled by effective potentials, adjusted to reproduce the overall physical properties 
  of the real system. This type of simulation is particularly used to study biological macromolecules such as proteins, where 
  atomic details are not always essential. Coarse-grained MD therefore represents a compromise between structural realism and numerical efficiency.
  In our case, we are studying different conformations of RNA using various crystallized structures (PDB). The ultimate goal is to find the 
  most stable conformation of the RNA. This research will depend on the mapping (grouping of atoms into beads to obtain a coarse-grained simulation) that we choose,
  and on a force field specific to our problem. Once the previous steps have been completed, we will then use a Monte Carlo method to explore the various 
  conformations until we obtain the most stable conformation, the one with the lowest energy. 
\end{multicols}

% ====================================================================================
% ============================ Explication des Fonctions =============================
% ====================================================================================
\chapter{Fonctions}

\subsection{\texttt{get\_arguments}}

Cette fonction a pour objectif de récupérer l'argument '--mapping' qui peut etre placé dans ligne d'execution du code afin de réalisé le mapping de tous les 
fichiers PDB d'un dossier. Si '--mapping' n'est pas présent alors le mapping ne sera pas éffectué. Le nouveau mapping remplace le précédent.

\section{Fonctions utilisées pour le Mapping}

\subsection{\texttt{check\_purine}}

Cette fonction sert a vérifier si la sous-structure observé dans l'ARN correspond a une purine ou a une pirimidine. En effet, le mapping implique des atomes des bases
azotées ainsi il est important de faire la différences entre les deux.

\subsection{\texttt{centre\_de\_masse}}

Cette fonction à pour objectif de récupérer les coordonnées d'un groupe d'atomes que l'on veut regrouper en une seule bead afin de déterminer la position de cette 
bead en calculant le centre de masse du groupe d'atomes.

\subsection{\texttt{mapping\_RIA}}


\subsection{\texttt{formatage}}
\subsection{\texttt{MAPPING}}

\section{Fonctions utilisées pour le Champ de force}

\subsection{\texttt{distance}}
\subsection{\texttt{E\_Lennard\_Jones}}
\subsection{\texttt{E\_elastique}}
\subsection{\texttt{compute\_energy}}

\section{Fonctions pour la méthode de Monte Carlo}

\subsection{\texttt{MMC}}

\section{Fonctions de plot et d'analyse diverses}

\subsection{\texttt{analyse\_T\_pas}}

\subsection{\texttt{analyse\_profil}}
% ====================================================================================
% ==================================== Le Mapping ====================================
% ====================================================================================

\chapter{Le Mapping}

En effet, un ARN se constitue d'une chaine 
très similaire à l'ADN. L'ADN est une double hélice constituée de deux chaines en opposition composé chacune d'un phosphore, d'un deoxiribose et d'une base azotée


% ====================================================================================
% ========================== Explication du champ de force ===========================
% ====================================================================================

\chapter{Le champ de Force}

\section{Les différents potentiels}

\section{Les paramètres associés}

% ====================================================================================
% ============================ Explication des Résultats =============================
% ====================================================================================
\chapter{Résultats}


% ====================================================================================
% =================================== Discussions ====================================
% ====================================================================================

\chapter{Discussion}

\end{document}
