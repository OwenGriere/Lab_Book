\documentclass[12pt,a4paper]{report}

% ========= Préambule importé depuis help/base.tex (extrait avant \begin{document}) =========

% =============== Encodage & langue ===============
\usepackage[T1]{fontenc}
\usepackage[utf8]{inputenc}
\usepackage[french]{babel}
\usepackage{lmodern}
\usepackage{amsmath}
\usepackage{microtype}
\usepackage{multicol}
\usepackage{float} 

% =============== Mise en page ===============
\usepackage[a4paper,margin=1.5cm, bottom=2.2cm]{geometry}
\usepackage{setspace}
\setlength{\headheight}{15pt}
\setlength{\footskip}{12mm}  
\onehalfspacing

% =============== Images & couleurs ===============
\usepackage{graphicx}
\graphicspath{{figures/}}
\usepackage[dvipsnames]{xcolor}
\usepackage{amsmath,amssymb,amsfonts}

\usepackage{tikz}
\DeclareRobustCommand{\legenddot}[2][3pt]{%
  \tikz[baseline=-0.6ex]{\draw[draw=black,fill=#2] (0,0) circle (#1);}%
}
% =============== En-têtes / pieds de page ===============
\usepackage{fancyhdr}
\pagestyle{fancy}
\fancyhf{}
\renewcommand{\headrulewidth}{0pt}
\lhead{} \chead{} \rhead{}          
\cfoot{\thepage}

% =============== Liens cliquables & signets PDF ===============
\usepackage{hyperref}
\hypersetup{
  pdfauthor   = {Owen GRIERE},
  pdftitle    = {Rapport de Stage - CRCT - 2025},
  pdfsubject  = {Rapport de Stage},
  pdfcreator  = {LaTeX},
  colorlinks  = true,
  linkcolor   = teal,
  urlcolor    = teal,
  citecolor   = teal
}
\usepackage{bookmark}
\usepackage{wrapfig}
\usepackage{array}
\usepackage{enumitem}
\usepackage{tabularx}
\usepackage{etoolbox} 
\usepackage{caption, subcaption}
\captionsetup{font={scriptsize,it}}

%=============== Bibliographie ===============

\makeatletter
\newcommand{\tocnopage}[2]{%
  \@ifpackageloaded{hyperref}{%
    \addtocontents{toc}{\protect\contentsline{#1}{#2}{}{}}
  }{%
    \addtocontents{toc}{\protect\contentsline{#1}{#2}{}}
  }%
}
\usepackage[section]{placeins}
\usepackage{titlesec}
\usepackage[backend=biber,style=numeric]{biblatex}
\addbibresource{references/references.bib}

%=============== Glossaire ===============

\usepackage[acronym]{glossaries}
\makeglossaries
\setacronymstyle{long-short}

%=============== Format des titres ===============

\makeatother
\setlist{nosep}

\titleformat{\chapter}{\normalfont\huge\bfseries}{\thechapter}{1em}{}
\titlespacing*{\chapter}{0pt}{-10pt}{20pt}

\titleformat{\section}[block]{\Large\bfseries}{\thesection}{0.8em}{}      
\titlespacing*{\section}{0pt}{2ex plus .2ex}{1ex plus .1ex}               

\titleformat{\subsection}[block]{\large\bfseries}{\thesubsection}{0.8em}{} 
\titlespacing*{\subsection}{0pt}{1.5ex plus .2ex}{0.7ex plus .1ex}


%=======================================
%============ Page de garde ============
%=======================================


\begin{document}


\begin{titlepage}
  % --- Bandeau de logos en haut de page ---
  \begin{center}
    \setlength{\tabcolsep}{8pt} % espace horizontal entre logos
    \renewcommand{\arraystretch}{1.0} % hauteur des lignes
    \begin{tabular*}{\textwidth}{@{\extracolsep{\fill}}ccccccc@{}}
      \includegraphics[height=3.5cm]{logo_eidd.png} \\
    \end{tabular*}
  \end{center}

  \vspace*{3cm} % espace après les logos

  % --- Titre principal ---
  \centering
  {\fontsize{44}{52}\selectfont\bfseries
  TP Modélisation MD Gros Grain\par}

  \vspace*{\fill}

  % --- Ligne et informations ---
  \noindent\rule{\textwidth}{1pt}
  \vspace{0.6em}
  {\raggedright
    \large Auteur : Owen GRIERE\par
    \large Date : \today\par
  }

\end{titlepage}


\cleardoublepage
\pdfbookmark[section]{Sommaire}{toc} 
\tableofcontents
\newpage

% ====================================================================================
% =================================== Introduction ===================================
% ====================================================================================
\chapter{Introduction}

\begin{multicols}{2}
  La simulation MD est une méthode numérique permettant de suivre le mouvement de particules en résolvant les équations de Newton 
  à partir de forces dérivées de potentiels d’interaction. Dans le cas d’une modélisation gros grain (coarse-grain), plusieurs atomes sont regroupés en 
  une seule entité appelée « beads », afin de simplifier la description du système. Cette approche réduit considérablement le nombre de 
  degrés de liberté et donc le coût computationnel. Elle permet ainsi d’accéder à des échelles de temps et d’espace plus grandes que celles atteignables par 
  les simulations atomistiques classiques. Les interactions entre les billes sont modélisées par des potentiels effectifs, ajustés pour reproduire les propriétés 
  physiques globales du système réel. Ce type de simulation est particulièrement utilisé pour étudier les macromoldécules biologiques comme les protéines où les 
  détails atomiques ne sont pas systématiquement indispensables. La MD gros grain constitue donc un compromis entre réalisme structurel et efficacité numérique.
  Dans notre cas nous étudions différentes conformation d'un ARN grâce à diverses structures crystalisées (PDB). L'objectif final étant de trouver la conformation 
  la plus stable de l'ARN. Cette recherche dépendra du mapping (regroupement de groupes d'atomes en beads pour obtenir une simulation gros grain) que nous choisirons,
  d'un champ de force spécifique a notre problème. Une fois les étapes précédentes accomplies nous utiliserons alors une methode de Monte Carlo pour explorer les diverses 
  conformations jusqu'à obtenir la conformation la plus stable, donc la plus basse en énergie. 
\end{multicols}

\vspace{1em}

\begin{multicols}{2}
  MD simulation is a numerical method for tracking particle motion by solving Newton's equations 
  based on forces derived from interaction potentials. In coarse-grain modeling, several atoms are grouped together into 
  a single entity called “beads” in order to simplify the description of the system. This approach significantly reduces the number of 
  degrees of freedom and therefore the computational cost. It thus allows access to larger time and space scales than those achievable by 
  classical atomistic simulations. The interactions between the beads are modeled by effective potentials, adjusted to reproduce the overall physical properties 
  of the real system. This type of simulation is particularly used to study biological macromolecules such as proteins, where 
  atomic details are not always essential. Coarse-grained MD therefore represents a compromise between structural realism and numerical efficiency.
  In our case, we are studying different conformations of RNA using various crystallized structures (PDB). The ultimate goal is to find the 
  most stable conformation of the RNA. This research will depend on the mapping (grouping of atoms into beads to obtain a coarse-grained simulation) that we choose,
  and on a force field specific to our problem. Once the previous steps have been completed, we will then use a Monte Carlo method to explore the various 
  conformations until we obtain the most stable conformation, the one with the lowest energy. 
\end{multicols}

% ====================================================================================
% ============================ Explication des Fonctions =============================
% ====================================================================================
\chapter{Fonctions}

\subsection{\texttt{get\_arguments}}

Cette fonction a pour objectif de récupérer l'argument '--mapping' qui peut etre placé dans ligne d'execution du code afin de réalisé le mapping de tous les 
fichiers PDB d'un dossier. Si '--mapping' n'est pas présent alors le mapping ne sera pas éffectué. Le nouveau mapping remplace le précédent.

\section{Fonctions utilisées pour le Mapping}

Dans cette section, nous allons introduire les fonctions permettant d'effectuer le mapping en gros grain des fichiers PDB qui sont à l'origine au niveau atomistiques.

\subsection{\texttt{check\_purine}}

Cette fonction sert a vérifier si la sous-structure observé dans l'ARN correspond a une purine ou a une pyrimidine. En effet, le mapping implique des atomes des bases
azotées ainsi il est important de faire la différences entre les deux.

\subsection{\texttt{centre\_de\_masse}}

Cette fonction à pour objectif de récupérer les coordonnées d'un groupe d'atomes que l'on veut regrouper en une seule bead afin de déterminer la position de cette 
bead en calculant le centre de masse du groupe d'atomes.

\subsection{\texttt{mapping\_RIA}}

Cette fonction sert à effectuer le “mapping gros grain” d’un nucléotide d’ARN en regroupant ses atomes en 5 “beads” représentatives. Elle identifie d’abord le type 
de base (purine ou pyrimidine), puis renomme et réindexe certains atomes clés. Trois atomes de la base seront renommé en Ni1, Ni2 et Ni3.
Le groupement phosphate (Pho) reste inchangé simplement renommé, et les atomes constituant le sucre qui lie la base azotée aux groupement phosphate est renommé en 
ribose (RIB). Les coordonnées du ribose sont remplacées par son centre de masse, calculé à partir des atomes du cycle sucre.
Chaque “bead” reçoit un identifiant unique dépendant du rang du nucléotide (sT).
Enfin, la fonction renvoie une nouvelle liste d’atomes représentatifs (4 ou 5 selon la présence du phosphate) correspondant au modèle gros grain de ce nucléotide.

\subsection{\texttt{formatage}}

Cette fonction sert a formater le resultat du mapping dans une structure de fichier de type PDB ou chaque element n'est pas séparé par une tabulation mais doit 
impérativement se situer dans un interval de colonnes.

\subsection{\texttt{MAPPING}}

Cette fonction réalise le mapping en parcourant tout le fichier PDB, et en executant les fonctions précédentes en itérant sur les nucléotides afin de réecrire un 
fichier de mapping pour chaque fichier PDB ouvert.

\section{Fonctions utilisées pour le Champ de force}

Dans cette section nous allons décrire les fonctions qui composent la construction du champ de force ainsi que le calcul de l'énergie libre de chaque conformation.

\subsection{\texttt{distance}}

Cette fonction calcul la distance euclidienne entre deux atomes i et j en trois dimensions.

\[
r_{ij} = \sqrt{(x_i - x_j)^2 + (y_i - y_j)^2 + (z_i - z_j)^2}
\]

\noindent
où :
\begin{itemize}
  \item $x_i, y_i, z_i$ sont les \textbf{coordonnées cartésiennes} de l’atome $i$,
  \item $x_j, y_j, z_j$ sont celles de l’atome $j$,
  \item $r_{ij}$ représente la \textbf{distance euclidienne} entre les deux atomes dans l’espace tridimensionnel.
\end{itemize}

\subsection{\texttt{E\_Lennard\_Jones}}

\begin{wrapfigure}[4]{l}{0.3\textwidth}
  \centering
  \includegraphics[width=0.30\textwidth]{lennard-jones.png}
  \caption{Potentiel de Lennard-Jones entre deux atomes ou beads non liés}
  \label{fig:L-J}
\end{wrapfigure}

\noindent \\ \\ Cette fonction calcule le potentiel de Lennard-Jones entre deux atomes non reliés par le squelette de la manières suivante :

\[
E_{\mathrm{LJ}}(i,j) = 4 \, \varepsilon_{ij} 
\left[
\left( \frac{\sigma_{ij}}{r_{ij}} \right)^{12}
- 
\left( \frac{\sigma_{ij}}{r_{ij}} \right)^{6}
\right]
\]

\vspace{6em}

où :
\begin{itemize}
  \item $\varepsilon_{ij} = \sqrt{\varepsilon_i \, \varepsilon_j}$ est la \textbf{profondeur du puits de potentiel}, représentant l’intensité de l’interaction attractive,
  \item $\sigma_{ij} = \tfrac{1}{2} (\sigma_i + \sigma_j)$ est la \textbf{distance caractéristique} à laquelle le potentiel est nul,
  \item $r_{ij}$ est la \textbf{distance} entre les atomes $i$ et $j$,
  \item le terme $\left( \tfrac{\sigma_{ij}}{r_{ij}} \right)^{12}$ modélise la \textbf{répulsion} à courte portée (chevauchement électronique),
  \item tandis que $\left( \tfrac{\sigma_{ij}}{r_{ij}} \right)^{6}$ décrit l’\textbf{attraction de van der Waals} à longue portée.
\end{itemize}
\newpage

\subsection{\texttt{E\_elastique}}

\begin{wrapfigure}{l}{0.3\textwidth}
  \centering
  \includegraphics[width=0.30\textwidth]{courbe-elast.png}
  \caption{Potentiel élastique entre deux atomes ou beads chimiquement liés}
  \label{fig:L-J}
\end{wrapfigure}

\noindent \\ Cette fonction calcule le potentiel de Lennard-Jones entre deux atomes reliés par le squelette de la manières suivante :

\[
E_{\text{élastique}}(i, j) = \frac{1}{2} \, k_{ij} \, (r_{ij} - r_{0,ij})^2
\]

où :
\begin{itemize}
  \item $k_{ij}$ est la \textbf{constante de raideur} associée à la paire d'atomes $(i,j)$, obtenue à partir de la table $K_{\mathrm{EL}}$,
  \item $r_{0,ij}$ est la \textbf{distance d’équilibre} entre les deux atomes, issue de la table $R0_{\mathrm{EL}}$,
  \item $(r_{ij} - r_{0,ij})$ représente l’\textbf{élongation} du lien,
  \item le facteur $\tfrac{1}{2}$ correspond à la \textbf{forme harmonique} de l’énergie selon la loi de Hooke.
\end{itemize}

\subsection{\texttt{E\_electrostatic}}

Cette fonction calcule le potentiel electrostatique entre deux atomes non reliés par le squelette de la manière suivante :

\[
E_{\mathrm{elec}}(i,j) = \frac{1}{4\pi \varepsilon_0 \varepsilon_r} 
\cdot \frac{q_i \, q_j}{r_{ij}}
\]

où :
\begin{itemize}
  \item $q_i$ et $q_j$ sont les \textbf{charges électriques} des atomes $i$ et $j$,
  \item $r_{ij}$ est la \textbf{distance} séparant les deux atomes,
  \item $\varepsilon_0$ est la \textbf{perméabilité du vide},
  \item $\varepsilon_r$ est la \textbf{constante diélectrique relative} du milieu,
  \item $\dfrac{1}{4\pi \varepsilon_0 \varepsilon_r}$ est le \textbf{facteur de Coulomb}, qui module l’intensité de l’interaction selon le milieu considéré.
\end{itemize}


\subsection{\texttt{compute\_energy}}

Cette fonction calcule l’énergie libre d’une conformation moléculaire à partir d’un fichier d’entrée (au format type PDB). 
Elle commence par lire toutes les lignes du fichier et extrait uniquement celles correspondant aux atomes (`ATOM`), qu’elle stocke dans une liste. 
Ensuite, elle parcourt toutes les paires d’atomes pour évaluer leurs interactions dans une double boucle sans redondance. 
Si les deux atomes appartiennent à la même sous-structure et appartiennent au squelette (grâce la liste `bonds`), 
ou s’il s’agit de deux groupements phosphates voisins le long de la chaîne, la fonction calcule une énergie élastique. 
Dans tous les autres cas, elle évalue le potentiel electrostatique ainsi que l’interaction de type Lennard-Jones, représentant les forces de van der Waals 
entre beads non liés. La somme de toutes ces contributions fournit l’énergie libre U du système.


\section{Fonctions pour la méthode de Monte Carlo}

\subsection{\texttt{MMC}}

Comme vu dans le précédent TP cette fonction a simplement pour objectif de parcourir les différentes conformations de l'ARN qui se trouve être les différents fichiers
PDB. Cette fonction utilise l'algorithme de Monte Carlo pour etre plus efficace et trouver la conformation la plus stable.

\section{Fonctions de plot et d'analyse diverses}

\subsection{\texttt{analyse\_T\_pas}}

Cette fonction sert a plot une heatmap de la performance de la méthode de Monte Carlo en fonction de diverses valeur de Température et de nombre de pas, a chaque 
itérations nous réalisons un nombre choisit d'expériences. Dans le cas de la Figure~\ref{fig:no_infuence_T}, nous avons réalisé 1000 expériences de Monte Carlo pour chacune des valeurs 
de Température et de nombre de pas.

\begin{figure}[H]
  \centering
  \includegraphics[width=0.90\textwidth]{no_influence_T.png}
  \caption{Heatmap montrant la performance de la MMC pour différentes valeur de Température et de nombre de pas par MMC, le nombre d'expérience 
  a été fixé a 1000 pour pouvoir récupérer des statistiques plus fiables}
  \label{fig:no_infuence_T}
\end{figure}

\subsection{\texttt{analyse\_profil}}

\begin{figure}[H]
  \centering
  \includegraphics[width=0.80\textwidth]{profil_energy.png}
  \caption{Paysage energétique de toutes conformations disponibles de notre ARN (avec Plotly)}
  \label{fig:profil_energy}
\end{figure}

Cette fonction permet de voir comment est distribuer l'energie libre de chacune des conformations a travers un graphique intéractif Plotly où tout les fichiers 
sont dans le plan \(P=(X, Y)\) et l'energie associé aux conformations en Z.
Dans la Figure~\ref{fig:profil_energy} on a du mal a distinguer pour quel fichier est associé une telle energie mais nous pouvons observer le paysage global de 
l'energie libre.

\section{Fonctionnement global}

Pour articuler tout ces fonctions, j'utilise un main qui va executer le mapping (si besoin), calculer l'energie libre de chacune des structures et les placer 
dans un dictionnaire qui a associe une structure a son énergie libre totale. Ensuite, celui-ci execute la méthode de Monte Carlo pour trouver la meilleur conformation 
pas à pas. Enfin j'effectue l'analyse que je désire en appelant la fonction souhaité.

% ====================================================================================
% ==================================== Le Mapping ====================================
% ====================================================================================

\chapter{Le Mapping}

Un ARN se constitue d'une chaine très similaire à l'ADN. L'ADN est une double hélice constituée de deux chaines en opposition composé chacune d'un phosphore, 
d'un deoxiribose et d'une base azotée tandis que l'ARN est constitué d'un ribose complet. 
Donc notre ARN est une chaine de nucléotide que l'on peut réduire dans une modélisation gros grain grâce au fait que l'ARN est un schéma répétitif.

\noindent Je choisis comme mapping le modèle XIA qui propose de conserver le groupe phosphate tel quel comme maillon de la chaine, le cycle sucre est remplacé par une seule bead
la base azotée est remplacé par 3 beads formant un triangle

\noindent L'objectif du mapping de la base azotée est de conserver la forme de la base azotée qui est systématiquement plane. La géométrie du triangle permet de conserver, même 
en utilisant des beads, la structure plane de la base azotée. Le mapping que j'utilise est le suivant : \\

\begin{description}
  \item[- Ribose] \emph{cycle sucre transformé en une bead} 
  \item[- Pho] \emph{groupe phosphate transformé en une bead}
  \item[- Base azotée de type purine] \emph{trois atomes sont conservés}  
  \item[- Base azotée de type pyrimidine] \emph{trois atomes sont conservés}
\end{description}

\noindent \\ Ainsi, j'utilise des listes pour récupérer les atomes de chaque structure du nucléotide : \\ Les atomes récupéré pour le cycle sucre 
sont : O3' C3' HO2' O2' C2' C1' O4' C4' C5' O5' H05'

\noindent Je fais la différences entre les purines et pyrimidines dans mon mapping car elles n'ont pas la même structure et donc pas les mêmes propriétés méchanistiques.

\noindent La purine est composé de deux cycle contrairement a la pyrimidine qui n'en possède qu'un.
Tout d'abord nous allons regarder le mapping pour une pyrimidine grace au logiciel PyMol. Pour une base azotée de ce type les atomes récupéré sont : N9, C2 et C6

\begin{figure}[H]
  \centering
  \begin{minipage}{0.48\textwidth}
    \centering
    \includegraphics[width=\textwidth]{pyrimidine.png}
    \caption{Visualisation en 3D d'une pyrimidine au sein d'un nucléotide}
    \label{fig:pyrimidine}
  \end{minipage}\hfill
  \begin{minipage}{0.48\textwidth}
    \centering
    \includegraphics[width=\textwidth]{pyrimidine_mapping.png}
    \caption{Visualisation en 3D d'une pyrimidine au sein d'un nucléotide où les atomes conservés pour le mapping sont modélisé en sphères magenta}
    \label{fig:pyrimidine-mapped}
  \end{minipage}
\end{figure}

\noindent \\ Ensuite, nous allons regarder le mapping d'une purine grace au logiciel PyMol. Pour une base azotée de ce type les atomes récupéré sont : N1, N3 et C5

\begin{figure}[H]
  \centering
  \begin{minipage}{0.48\textwidth}
    \centering
    \includegraphics[width=\textwidth]{purines.png}
    \caption{Visualisation en 3D d'une purine au sein d'un nucléotide}
    \label{fig:purine}
  \end{minipage}\hfill
  \begin{minipage}{0.48\textwidth}
    \centering
    \includegraphics[width=\textwidth]{purine_mapping.png}
    \caption{Visualisation en 3D d'une purine au sein d'un nucléotide où les atomes conservés pour le mapping sont modélisé en sphères jaunes}
    \label{fig:purine-mapped}
  \end{minipage}
\end{figure}

\noindent Le mapping complet est visible dans la Figure~\ref{fig:mapping}. Dans cette figure, on peut voir 

\begin{figure}[H]
  \centering
  \includegraphics[width=0.80\textwidth]{mapping_cartoon.png}
  \caption{Mapping complet de l'ARN ou les bead sont visibles sur une vue 'cartoon' de l'ARN depuis PyMol}
  \label{fig:mapping}
\end{figure}

% ====================================================================================
% ========================== Explication du champ de force ===========================
% ====================================================================================


\chapter{Le champ de Force}

\section{Les différents potentiels}

\section{Les paramètres associés}

% ====================================================================================
% ============================ Explication des Résultats =============================
% ====================================================================================
\chapter{Résultats}


% ====================================================================================
% =================================== Discussions ====================================
% ====================================================================================

\chapter{Discussion}

\end{document}
